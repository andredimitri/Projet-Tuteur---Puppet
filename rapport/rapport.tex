\part{Le Grid5000}
\chapter{Présentation du Grid5000}
Le Grid5000 est une grille de calcul destiné à la recherche scientifique. Le projet a vu le jour en 2003 et a pour but de promouvoir la recherche sur les grilles informatiques en France. Le projet est aujourd'hui composé de 1200 noeuds répartis sur 9 sites différents situés en France et au Luxembourg et interconnectés avec le réseau Réseau National de télécommunications pour la Technologie l'Enseignement et la Recherche (RENATER). L'objectif du Grid5000 est de permettre aux scientifiques d'effectuer des expériences dans le domaine des systèmes informatiques et des réseaux distribuées dans un environnement hétèrogene aussi proche de la réalité que possible.\n

\chapter{Les outils du Grid5000}
\section{OAR}
\section{Kadeploy}
\section{Taktuk}
\section{Les autres outils}
